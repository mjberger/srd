\section{Introduction}\label{sec:intro}
Cut cell meshes to solve hyperbolic problems 
are increasingly prevalent due to the ease 
of grid generation for complicated geometries. 
However the {\em small cell} problem is still an active area of research, and
a completely satisfactory solution has not yet been found.
The small cell problem can be explained as follows: explicit
finite volume schemes for hyperbolic problems are subject to a CFL constraint, i.e.,  they typically need to take a time step that is proportional to the mesh width for
stability. However cut cells can have volumes that are arbitrarily
smaller than the regular cells.  This forces the scheme to take overly restrictive time steps, even though the domain is predominantly composed of regular cells that would otherwise determine the stable time step. Special algorithms are needed to prevent this restriction.

The most commonly used stabilization algorithm is called flux
redistribution \cite{chern:colella,vof:colella}. The main idea is illustrated below in two space
dimensions for ease of notation.
The update of the solution average on cut cell $i,j$, $U^n_{i,j}$, is given by
\begin{equation*}
V_{i,j} U_{i,j} ^{n+1}  = V_{i,j} U_{i,j}^n  -  \Delta t \sum_{k \in \text{ faces}}A_k \mathbf{F}_k^* \cdot \mathbf{n}_k,
\end{equation*}
where $U^n_{i,j}$ is the solution average at time $t^n$ on cell $(i,j)$, $V_{i,j}$ is the cell's volume,  $\Delta t$ is the time step, $A_k$, $\mathbf{F}_k^*$, and $\mathbf{n}_k$ are the area, the numerical flux, and the outward facing normal on the $k$th face.
The above can also be written
\begin{equation*}
V_{i,j} U_{i,j} ^{n+1} = V_{i,j} U_{i,j}^n  +  \delta  M ,
\end{equation*}
where $\delta M$ is the change of mass on cell $(i,j)$ after one time step.
Instead of using the entire amount of the update in cell ${i,j}$, 
the cut cell only uses a fraction $\eta$ of it.  If the fraction $\eta$
is proportional to the cell's volume
fraction $V_{i,j}/V_{\text{full}}$, where $V_{\text{full}}$ is the volume of a whole Cartesian cell, the update should be stable. 
To maintain conservation, the rest of the update ($1-\eta)\delta M$
is given to the cell's neighbors.  
There are more recent additional steps that make the distribution 
more robust and accurate. For example, the difference between a stable
non-conservative update and the conservative update is what is
redistributed (see \cite{vof:colella} for details).
Flux redistribution has already been implemented for three dimensional
calculations due to its simplicity. However it is only first order
accurate at the cut cells.

Cell merging is most frequently the first solution that comes to mind for cut cell stabilization. It is conceptually simple, but 
we are not aware of any production codes that implement this in a fully
general, robust manner for complicated engineering geometries. 
The $h$-box method \cite{mjb-hel-rjl:hbox2,mjb-hel:hboxsimple}
is a second order accurate method at the cut cells. It extends the 
domain of dependence for the fluxes around a small cell in a 
special way that maintains stability by means of a cancellation
property. It  has not been extended to
three dimensions due to its complexity. 

A newer variation of cell merging is called cell linking \cite{cecereGiacomazzi,
KirkpatrickEtAl:2003, HuKhooAdamsHuang:2006,Chung:2006} 
This has simpler data structures and maintain the original grid. 
Many of these references tackle incompressible flow; some also include
three dimensional examples. Several references use a staggered mesh. 
In \cite{BalajiMenon:2016}, the authors improve the accuracy of cell linking,
with a third order accurate approach for viscous flow,  and fourth order for 
inviscid flow. 
Their version of cell linking uses a cluster of cells,
while still maintaining each cell in the mesh.  A high order
polynomial is fit to the cluster, and replaces the solution values in the
individual cells.  Our algorithm has a similar spirit to this, though the
details are very different. 

In \cite{shws:2011}, the authors make
some improvements to flux redistribution for viscous flow in moving
geometries. They introduce a
smooth cutoff function of cell size for when it is applied. They also
use a non-uniform weighting in the gradient stencil to avoid abrupt
changes, which can lead to oscillations in the solution. This is
especially important for moving geometries.

In \cite{May-Berger:JSC}, an implicit scheme was developed to
handle stability of cut cells, and was combined with an 
explicit method for the full cells. In this work we will focus on 
two explicit finite volume methods, and make modification necessary for their accuracy 
on cut cells.
Other approaches that have been proposed in the literature include
interpolation-based procedures, such as the mirror-cell method by Forrer
and Jeltsch \cite{article:FoJe98}, and a related ghost-fluid method by
Dadone and Grossman \cite{DadoneGrossman}.
There are also approaches based on finite difference schemes
\cite{SjogreenPetersson,MarcoBjorn}
and kinetic schemes \cite{Oksuzoglu:thesis,KeenKarni}.
However, since we are interested in methods
that preserve conservation we do not explore these alternatives further.

In this paper we propose a framework for a stabilization algorithm in
the spirit of flux redistribution (henceforth FRD). 
As with FRD, it is applied as a postprocessing
step, and is simple to implement. 
An unstable updates on all cells are performed
with a fixed $\Delta t$ using whatever the base finite volume scheme is, followed by a 
postprocessing step based on the
conserved state variables, not on the fluxes.
Hence we call it {\em state redistribution} (SRD).



