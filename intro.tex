\section{Introduction}\label{sec:intro}
Cut cell meshes to solve hyperbolic problems 
are increasingly prevalent due to the ease 
of grid generation for complicated geometries. 
However, these meshes lead to the {\em small cell} problem, which
is still an active area of research, and  
a completely satisfactory solution has not yet been found.
The small cell problem can be explained as follows: explicit
finite volume schemes for hyperbolic problems are subject to a CFL constraint, i.e.,  they typically need to take a time step that is proportional to the mesh width for
stability. However, cut cells can have volumes that are arbitrarily smaller than the regular cells.  This would force the scheme to take overly restrictive time steps, even though the domain is predominantly composed of regular cells that would otherwise determine the stable time step. Special algorithms are needed to prevent this restriction.

The most commonly used stabilization algorithm is called flux
redistribution \cite{chern:colella,vof:colella}. The main idea is illustrated below in two space
dimensions for ease of notation.
The update of the solution average 
$U^n_{i,j}$ on cut cell $(i,j)$  is
\begin{equation*}
V_{i,j} U_{i,j} ^{n+1}  = V_{i,j} U_{i,j}^n  -  \Delta t \sum_{k \in \text{ faces}}A_k \mathbf{F}_k^* \cdot \mathbf{n}_k,
\end{equation*}
Here, $\Delta t$ is the time step,
volume $V_{i,j}$ is the cell volume,
$A_k$, $\mathbf{F}_k^*$, and $\mathbf{n}_k$ are the area, the numerical flux, and the outward facing normal on the $k$th face, respectively.
The above can also be written
\begin{equation*}
V_{i,j} U_{i,j} ^{n+1} = V_{i,j} U_{i,j}^n  +  \delta  M_{i,j} ,
\end{equation*}
where $\delta M_{i,j}$ is the change in the conserved quantity 
on cell $(i,j)$ after one time step.
Instead of using the entire amount of the update in cell $(i,j)$, 
the cut cell only receives a fraction $\eta_{i,j}$ of it.  If the fraction $\eta_{i,j}$
is proportional to the cell's volume fraction $V_{i,j}/V_{\text{full}}$,
where $V_{\text{full}}$ is the volume of an uncut Cartesian cell, the update should be stable. 
To maintain conservation, the rest of the update ($1-\eta_{i,j})\delta M_{i,j}$
is given to the cell's neighbors.  
%There are more recent additional steps that make the distribution 
%more robust and accurate. For example, the difference between a stable non-conservative update and the conservative update is what is redistributed (see \cite{vof:colella} for details).
Flux redistribution has already been implemented for three dimensional
calculations due to its simplicity. However it is only first order accurate at the cut cells.

Cell merging \cite{icm-2003,Chung:2006}  is most frequently the first solution 
that comes to mind for cut cell stabilization. A cut cell is merged with
neighbors until a cell with sufficiently large volume for a stable time
step is obtained. 
It is conceptually simple, but 
we are not aware of any production codes that implement this in a fully
general, robust manner for complicated engineering geometries. 
The $h$-box method \cite{mjb-hel-rjl:hbox2,mjb-hel:hboxsimple}
is a second order accurate method at the cut cells. It extends the 
domain of dependence for the fluxes around a small cell in a 
special way that maintains stability by means of a cancellation
property. It  has not been extended to
three dimensions due to its complexity. 

A newer variation of cell merging is called cell linking \cite{cecereGiacomazzi,
KirkpatrickEtAl:2003, HuKhooAdamsHuang:2006,Chung:2006}.
This has simpler data structures and maintains the original grid. 
%Many of these references tackle incompressible flow; some also include three dimensional examples. Several references use a staggered mesh. 
In \cite{BalajiMenon:2016}, the authors improve the accuracy of cell linking,
with a third order accurate approach for viscous flow,  and fourth order for 
inviscid flow. 
Their version of cell linking uses a cluster of cells, while still
maintaining each cell in the mesh.  A high order polynomial is fit to
the cluster, and replaces the solution values in the individual cells.
Our state redistribution algorithm has a similar spirit to this, though the 
details are very different. 
%In \cite{shws:2011}, the authors make
%some improvements to flux redistribution for viscous flow in moving
%geometries. They introduce a
%smooth cutoff function of cell size for when it is applied. They also
%use a non-uniform weighting in the gradient stencil to avoid abrupt
%changes, which can lead to oscillations in the solution. This is
%especially important for moving geometries.

%In \cite{May-Berger:JSC}, an implicit scheme was developed to
%handle stability of cut cells, and was combined with an 
%explicit method for the full cells. In this work we will focus on 
%two explicit finite volume methods, and make modification necessary for their accuracy 
%on cut cells.
%Other approaches that have been proposed in the literature include
%interpolation-based procedures, such as the mirror-cell method by Forrer
%and Jeltsch \cite{article:FoJe98}, and a related ghost-fluid method by
%Dadone and Grossman \cite{DadoneGrossman}.
%There are also approaches based on finite difference schemes
%\cite{SjogreenPetersson,MarcoBjorn}
%and kinetic schemes \cite{Oksuzoglu:thesis,KeenKarni}.
%However, since we are interested in methods
%that preserve conservation we do not explore these alternatives further.
Two other approaches in the literature include the use of an implicit
scheme on the cut cells combined with an explicit scheme elsewhere
\cite{May-Berger:JSC}, and a novel flux interpolation scheme which has
the added advantage of being dimensionally split, so easier to
implement \cite{gokhaleNikosKlein:2018}. Our new approach is rather
different from these, however, and we do not pursue these directions
further.


In this paper we propose a stabilization algorithm in
the spirit of flux redistribution. Similar to flux redistribution, state
redistribution is applied as a postprocessing step and is simple to implement,
for the second order accurate case. We perform an unstable update on all 
cells with a fixed $\Delta t$ using a base finite volume scheme, followed
by a postprocessing step based on the conserved state variables, not on the
fluxes.  It is for this reason that we call it state redistribution
(SRD).  Our
approach is fully conservative and can be generalized to high order accuracy,
albeit with  more complexity. The key insight over cell merging was to
recognize that we could maintain both conservation and accuracy using a
weighted convex combination of solution values that takes into account the 
number of overlapping neighborhoods on each cell in the base cut cell
grid.

The important difference between state redistribution and cell merging is
that SRD supports overlapping neighborhoods, and cell merging does not.
As a result, cell merging can be difficult to implement robustly
in three dimensions, since there are many different, possibly
incompatible ways to create non-overlapping merged cells.  State
redistribution does not suffer from this difficulty, and its extension to
three dimensions is straightforward.

The next section illustrates state redistribution  in one space dimension
on a model problem, to give a more intuitive idea without all the details and 
notation of the second order accurate  case.
Section \ref{sec:basefv} discusses the evolution schemes on cut cell meshes
to which SRD is applied.
Section \ref{sec:srdAlg} describes the second order accurate
algorithm in two space dimension.  Computational experiments with the
two-dimensional Euler equations  are presented in
Section \ref{sec:compResults}, and conclusions in Section \ref{sec:conc}. 
%Our SRD  algorithm has been extended
%to higher order. This is described in the Appendix, where third and fourth order
%convergence results are presented.
We see no reason that this algorithm cannot be extended to higher order
accuracy, and have already started implementing the third and fourth
order accurate cases.

