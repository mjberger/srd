\section{Higher-Order State Redistribution Algorithm}\label{sec:ho}

The higher order version is essentially the same as the second order
version, but with a larger stencil so a higher order polynomial can be
reconstructed for each merging neighborhood. Of course, the base method
has to be higher order too. We use a 3rd order TVD Runge Kutta scheme
for the base method.  The spatial accuracy is improved by using a
quadratic or cubic reconstruction, taking care to translate
accurately between conservtive and primitive variables.  Higher
order polynomials are also used for reconstruction on the merged cells. 
The stability limit for the 3rd order TVD Runge-Kutta scheme with
quadratic reconstruction is
\begin{equation}
ANDREW - What is it?
\end{equation}
Again this is fairly standard, so we go right
to the description of the higher-order SRD algorithm.

\begin{enumerate}
\item 
\textbf{Compute the solution average on each merging tile}

This step is the same as the second order case.
Again using $\bar{U}_k$ to represent the provisionally updated 
solution after one step or stage, form merged solutions 
\begin{equation}\label{eq:q_avg1}
     \widehat{Q}_{ij} =  \frac{1}{ V_m}_{ij} \, \sum_{k \in
     M_i}\frac{V_k}{N_k} \bar U_k.
\end{equation}
\noindent where $M_i$ is again the set of indices of cells in the 
merging neighborhood, and the volume of the merged tile 
\begin{equation}\label{eq:modV}
{V_m}_{ij} = \sum_{k \in M_i}\frac{V_k}{N_k} .
\end{equation}

\item \textbf{Determine a polynomial reconstruction $\hat q_i(x)$ on 
each merging tile of the form}
\begin{equation}\label{eq:q}
\begin{aligned}
    \hat q_(x,y) = \hat Q_{m} + \hat \sigma_{m,x}\frac{x-\hat x_m}{\Delta x} +  \hat \sigma_{m,y}\frac{y-\hat y_m}{\Delta y} + \frac{1}{2}\hat \delta_{m, xx}\left[ \frac{(x - \hat x_m)^2 }{\Delta x^2} - \hat S_{m,xx}\right]\\
	    +\hat \delta_{m, xy}\left[ \frac{(x - \hat x_m) (y - \hat y_m) }{\Delta x \Delta y} - \hat S_{m,xy}\right] + \frac{1}{2}\hat \delta_{m, yy}\left[ \frac{(y - \hat y_m)^2 }{\Delta y^2} -  \hat S_{m,yy}\right]\\
	    + \frac{1}{6}\hat\gamma_{m, xxx}\left[ \frac{(x -  \hat x_m)^3 }{\Delta x^3} -  \hat S_{m,xxx}\right] + \frac{1}{2}\hat \gamma_{m, xxy}\left[ \frac{(x - \hat x_m)^2 (y -  \hat y_m) }{\Delta x^2 \Delta y} -  \hat S_{m,xxy}\right]\\
	     + \frac{1}{2}\hat \gamma_{m, xyy}\left[ \frac{(x -  \hat x_m) (y -  \hat y_m)^2 }{\Delta x \Delta y ^2} -  \hat S_{m,xyy}\right]+ \frac{1}{6}\hat \gamma_{m, yyy}\left[ \frac{(y -  \hat y_m)^3 }{\Delta y^3} -  \hat S_{m,yyy}\right],
\end{aligned}
\end{equation}
$\hat x_m$, $\hat y_m$ is the weighted centroid of the merging
neighborhood for cell $i,j$. 
The $\hat \sigma_{m,x}$, $\hat \sigma_{m,y}$, $\hat \delta_{m,xx}$, 
$\hat \delta_{m,xy}$, $\hat \delta_{m,yy}$, $ \hat \gamma_{m,xxx}$, 
$\hat  \gamma_{m,xxy}$, $\hat  \gamma_{m,xyy}$, $\hat  \gamma_{m,yyy}$ 
are the first, second, and third derivatives. 
The $ \hat S_{m,xx}$, $\hat S_{m,xy}$,  $\hat S_{m,yy}$ are geometric
constants that make it easier to maintain conservation when written in
this form.  The reconstruction satisfies
\begin{equation}\label{eq:qi}
\frac{1}{V_m}\sum_{k \in M_j}\frac{1}{N_k}\int_{\Omega_k} \hat q_i(x)~d\Omega_k = \hat Q_j \quad \forall j \in R_i,
\end{equation}
where $R_i$ is the set of indices of neighborhoods used for reconstruction 
on merging neighborhood $i$ of the form


\item \textbf{Set the final solution at time $t^{n+1}$}
	\begin{equation}\label{eq:final_update}
	U^{n+1}_i =  \frac{1}{V_i}\sum_{k \in M_{i}}\frac{1}{N_i}\int_{\Omega_i} \hat q_k(x)~d\Omega_i,
	\end{equation}
	where $M_i$ is a set of merging tile indices to which cell $i$ belongs.
\end{enumerate}

\subsection{Limiting on merging neighborhoods}


\subsection{Conservation}\label{sec:cons}
The total mass of the numerical solution at $t^{n+1}$ is
\begin{equation}\label{eq:total_mass}
\mathcal{M}^{n+1} = \sum^N_{i=1} h_i U^{n+1}_i.
\end{equation}
From the general form of the state redistribution algorithm 
in \eqref{eqn:final_update}, 
this can also be written as a sum of mass contributions from each 
merging neighborhood, i.e.,
\begin{equation}\label{eq:total_mass2}
\mathcal{M}^{n+1} = \sum^N_{i=1} \hat{\mathcal{M}}_i,
\end{equation}
where the mass contribution of merging neighborhood $i$ is
\begin{equation}\label{eq:mi}
\hat{\mathcal{M}}_i = \sum_{k \in M_i}\frac{1}{N_k} \int_{\Omega_k}\hat q_i(x) ~d\Omega_k,
\end{equation}
and $\hat q_i(x)$ is that neighborhood's polynomial reconstruction.  
From \eqref{eq:qi}, \eqref{eq:mi} becomes
\begin{equation}\label{eq:mi1}
\hat{\mathcal{M}}_i = \hat Q_i \hat V_i.
\end{equation}
% \subsection{First order algorithm}
% When $\hat q_i(x)$ is constant, i.e. $\hat q_i(x) = \hat Q_i$ by \eqref{eq:q_avg} and Step 2 in Section \ref{sec:first_order}, \eqref{eq:mi} becomes 
% Substituting \eqref{eq:mi} into \eqref{eq:total_mass2}, we have
% \begin{equation}\label{eq:mi1}
% \hat{\mathcal{M}}_i = \hat Q_i \hat V_i.
% \end{equation}
% by \eqref{eq:q_avg} in the first order algorithm, by \eqref{eq:pq2} in the second order algorithm, and by \eqref{eq:q2} in the third order algorithm.
From \eqref{eq:modV} and \eqref{eq:q_avg1}, \eqref{eq:mi1} becomes
\begin{equation}\label{eq:mi2}
\hat{\mathcal{M}}_i = \sum^{M_i}_{k = m_i}\frac{h_k}{N_k} \hat U_{k}.
\end{equation}
Substituting \eqref{eq:mi2} into \eqref{eq:total_mass2}, the mass at $t^{n+1}$ is
$$
\mathcal{M}^{n+1} = \sum^{N}_{i=1} h_i \hat U_i,
$$
which shows that $\mathcal{M}^{n+1}  = \mathcal{M}^{n} $.

	
