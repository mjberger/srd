\section{High-Order State Redistribution Algorithm}\label{sec:ho}

The high order version is essentially the same as the second order
version, but with a larger stencil so a higher order polynomial can be
reconstructed for each merging neighborhood. Of course, the base method
has to be higher order too. We use a third or fourth order TVD Runge Kutta scheme
for the base method.  The spatial accuracy is improved by using a
quadratic or cubic reconstruction, taking care to translate
accurately between conservative and primitive variables.  Higher
order polynomials are also used for reconstruction on the merged cells. 
The stability limit for the third order TVD Runge-Kutta scheme with
quadratic reconstruction is
\begin{equation}
ANDREW - What is it?
\end{equation}
Again this is fairly standard, so we go right to the description of the higher-order SRD algorithm.

\begin{enumerate}
\item
{\bf Compute one time step or stage of the base finite volume scheme.}  

$$
\widehat{U} = U^n + \Delta t L(U^n),
$$
where $L$ is the finite volume scheme described in section \ref{sec:basefv}.
\item 
\textbf{Compute the solution average on each merging tile}

This step is the same as the second order case.
Again using $\hat{U}_k$ to represent the provisionally updated 
solution after one step or stage, form merged solutions 
\begin{equation}\label{eq:q_avg1}
     \widehat{Q}_{i,j} =  \frac{1}{ \hat{V}_{i,j}} \,  \sum_{k \in M_{i,j} }\frac{V_{k}}{N_{k}} \hat{U}_{k},
\end{equation}
\noindent where $M_{i,j}$ is again the set of indices of cells in the 
merging neighborhood, and the weighted volume of the merged tile 
\begin{equation}\label{eq:modV}
\hat{V}_{i,j} = \sum_{k \in M_{i,j} }\frac{V_{k}}{N_{k}} .
\end{equation}

\item \textbf{Determine a polynomial reconstruction $\hat q_{i,j}(x)$ on 
each merging tile of the form}
% \begin{equation}\label{eq:q}
% \begin{aligned}
%     \hat q_(x,y) = \hat Q_{m} + \hat \sigma_{m,x}\frac{x-\hat x_m}{\Delta x} +  \hat \sigma_{m,y}\frac{y-\hat y_m}{\Delta y} + \frac{1}{2}\hat \delta_{m, xx}\left[ \frac{(x - \hat x_m)^2 }{\Delta x^2} - \hat S_{m,xx}\right]\\
% 	    +\hat \delta_{m, xy}\left[ \frac{(x - \hat x_m) (y - \hat y_m) }{\Delta x \Delta y} - \hat S_{m,xy}\right] + \frac{1}{2}\hat \delta_{m, yy}\left[ \frac{(y - \hat y_m)^2 }{\Delta y^2} -  \hat S_{m,yy}\right]\\
% 	    + \frac{1}{6}\hat\gamma_{m, xxx}\left[ \frac{(x -  \hat x_m)^3 }{\Delta x^3} -  \hat S_{m,xxx}\right] + \frac{1}{2}\hat \gamma_{m, xxy}\left[ \frac{(x - \hat x_m)^2 (y -  \hat y_m) }{\Delta x^2 \Delta y} -  \hat S_{m,xxy}\right]\\
% 	     + \frac{1}{2}\hat \gamma_{m, xyy}\left[ \frac{(x -  \hat x_m) (y -  \hat y_m)^2 }{\Delta x \Delta y ^2} -  \hat S_{m,xyy}\right]+ \frac{1}{6}\hat \gamma_{m, yyy}\left[ \frac{(y -  \hat y_m)^3 }{\Delta y^3} -  \hat S_{m,yyy}\right],
% \end{aligned}
% \end{equation}
\begin{equation}\label{eq:q}
\begin{aligned}
    \hat q_{i,j} (x,y) = \widehat{Q}_{i,j} +  \sum_{1 \leq |\alpha| \leq p}  \frac{1}{\alpha!} (\partial^{\alpha} \hat q) [h^{\alpha}-\hat S_{\alpha}]
\end{aligned}
\end{equation}
where $\alpha = (\alpha_1, \alpha_2)$, $|\alpha| = \alpha_1 + \alpha_2$, $h^{\alpha} = (x-\hat x_m)^{\alpha_1}(y-\hat y_m)^{\alpha_2}$, $\partial^{\alpha} = \partial^{\alpha_1}\partial^{\alpha_2}$,
and $\hat x_m$, $\hat y_m$ is the weighted centroid of the merging
neighborhood for cell $i,j$. 
The $ \hat S_{\alpha}$ is a geometric
constant that make it easier to maintain conservation when written in
this form.  The reconstruction satisfies in a least squares sense
\begin{equation}\label{eq:qi}
\frac{1}{\hat{V}_{k}}\sum_{k' \in M_{k}}\frac{1}{N_{k'}}\int_{\Omega_{k'}} \hat q_{i,j}(x)~d\mathbf{x} = \hat Q_{k} \quad \forall k \in R_{i,j},
\end{equation}
where $R_{i,j}$ is the set of indices of neighborhoods used for reconstruction 
on merging neighborhood $i$ of the form


\item \textbf{Set the final solution at time $t^{n+1}$}
	\begin{equation}\label{eq:final_update}
	U^{n+1}_{i,j} =  \frac{1}{V_{i,j}}\sum_{k \in W_{i,j}}\frac{1}{N_{i,j}}\int_{\Omega_{i,j}} \hat q_{k}(x)~d\mathbf{x} ,
	\end{equation}
	where $W_{i,j}$ is a set of merging neighborhoods to which cell $i,j$ belongs.
\end{enumerate}


\subsection{Conservation}\label{sec:cons}
The total mass of the numerical solution at $t^{n+1}$ is
\begin{equation}\label{eq:total_mass}
\mathcal{M}^{n+1} = \sum_{i,j} h_{i,j} U^{n+1}_{i,j}.
\end{equation}
From the general form of the state redistribution algorithm 
in \eqref{eq:final_update}, this can also be written as a sum of mass contributions from each 
merging neighborhood, i.e.,
\begin{equation}\label{eq:total_mass2}
\mathcal{M}^{n+1} = \sum_{i,j} \hat{\mathcal{M}}_{i,j},
\end{equation}
where the mass contribution of merging neighborhood $(i,j)$ is
\begin{equation}\label{eq:mi}
\hat{\mathcal{M}}_{i,j} = \sum_{k \in M_{i,j}}\frac{1}{N_{k}} \int_{\Omega_{k}}\hat q_{k}(x) ~d\mathbf{x},
\end{equation}
and $\hat q_{k}(x)$ is that neighborhood's polynomial reconstruction.  
From \eqref{eq:qi}, \eqref{eq:mi} becomes
\begin{equation}\label{eq:mi1}
\hat{\mathcal{M}}_{i,j} = \hat Q_{i,j} \hat V_{i,j}.
\end{equation}
% \subsection{First order algorithm}
% When $\hat q_i(x)$ is constant, i.e. $\hat q_i(x) = \hat Q_i$ by \eqref{eq:q_avg} and Step 2 in Section \ref{sec:first_order}, \eqref{eq:mi} becomes 
% Substituting \eqref{eq:mi} into \eqref{eq:total_mass2}, we have
% \begin{equation}\label{eq:mi1}
% \hat{\mathcal{M}}_i = \hat Q_i \hat V_i.
% \end{equation}
% by \eqref{eq:q_avg} in the first order algorithm, by \eqref{eq:pq2} in the second order algorithm, and by \eqref{eq:q2} in the third order algorithm.
From \eqref{eq:modV} and \eqref{eq:q_avg1}, \eqref{eq:mi1} becomes
\begin{equation}\label{eq:mi2}
\hat{\mathcal{M}}_{i,j} = \sum_{k \in M_{i,j} }\frac{h_{k}}{N_{k}} \hat U_{k}.
\end{equation}
Substituting \eqref{eq:mi2} into \eqref{eq:total_mass2}, the mass at $t^{n+1}$ is
$$
\mathcal{M}^{n+1} = \sum_{i,j} h_{i,j} \hat U_{i,j},
$$
which shows that $\mathcal{M}^{n+1}  = \mathcal{M}^{n} $.

	
