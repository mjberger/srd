\section{Some analysis in one dimension}\label{sec:theory}

For insight into some properties of our method
we study some a one dimensional model problem for stability and
accuracy on uniform grids.
We solve
$$
u_t + a \, u_x = 0,
$$
with periodic boundary conditions.  
The uniform grid is composed of $N$ cells with size $h = \frac{1}{N}$.
The merging neighborhoods are composed of an odd number cells, 
centered on cell $i$, i.e. cell $i$'s tile with three cells  
contains cells $i-1$, $i$, and $i+1$ .
The nonuniform grids are composed of $N$ randomly generated cells.
This more closely mimics the situation in two dimensions along the cut cell
embedded boundary, especially when using the $3\times3$ merging neighborhood.
This is not a good approximation of the normal merging neighborhood.
Nevertheless it is useful to study some theoretical properties to gain
intuition.
%(figure \ref{fig:overlapping1D}).
%\begin{figure}
%    \centering
%    \includegraphics[width = 0.8\textwidth]{figs/overlapping1D.pdf}
%    \caption{Overlapping neighborhoods on the uniform grid composed of three cells of width $k=1$.}
%    \label{fig:overlapping1D}
%\end{figure}
%AG  THE MERGING NHOOD CONTAINS CELLS TO THE LEFT AND RIGHT UNTIL  EACH SIDE IS
%DX/2 OR DX?
%OR DO WE USE YOUR REPEATING RANDON TILE OR A COMBO.
%TOOK OUT SIMPLE FIGURE OF UNIFORM MESH, BETTER TO USE RANDOM MESH EXAMPLE


\subsection{Truncation error}
The truncation error is
$$
\tau = \frac{u^{n+1}- L(u^n)}{\Delta t},
$$
where $u^n$ is a grid function of the exact solution at time $t^n$, 
and $L$ is the fully discrete finite volume update, i.e., a Runge Kutta scheme in time coupled with a finite volume spatial discretization stabilized with the state redistribution method.
When the scheme is stable, the truncation error can describe the 
order of accuracy of the finite volume method.
On uniform grids, we observe that that the truncation error 
is $||\tau|| = \mathcal{O}(h^{p+1})$.  
However, on nonuniform grids, we observe that the truncation error loses an order of accuracy, i.e., $||\tau|| = \mathcal{O}(h^{p})$.  
This is a well-known phenomenon for schemes on nonuniform grids \cite{}.
Nevertheless, we observe in the numerical examples in 
Section \ref{sec:compResults} demonstrate that the global error does not lose an order of
accuracy.

\subsection{Stability}
%We look at two stability properties in this section.
We note that the finite volume method stabilized with the state redistribution method is not monotone, total variation diminishing, and finally the numerical solution will not in general satisfy a maximum principle even with limiting.
This however does not preclude us from studying the $L_2$ stability of the numerical solution on regular, periodic grids.

In this section, we confirm the intuition that the maximum allowable time step given by von Neumann analysis increases with the size of the merging neighborhoods.  
%Next, we show that the error in the numerical solution increases with both the 
%size of the merging neighborhood and the size of the reconstruction 
%stencil on the tile. 
%This tradeoff helps determine the best tile size.
%(MADE IT UP - DOES IT? WOULD BE NICE TO SAY)
This section uses the TVD RK scheme as the base scheme and and we restrict
attention to uniform grids.
The finite volume update after one forward Euler time step can be written in 
matrix form
\begin{equation}\label{eq:L}
    \hat U = (I + \Delta t D)U^n,
\end{equation}
where $D$ is the finite volume approximation of $-a\frac{\partial u}{\partial x}$, and $\hat U = [\hat U_1, \hat U_2, \hdots, \hat U_N]$.
%The next step of the SRD method can be written in matrix form
%\begin{equation}\label{eq:merged_tile}
%\hat Q = T\hat U,
%\end{equation}
%where $\hat Q = [\hat Q_1, \hat Q_2, \hdots, \hat Q_N]$, and the 
%matrix $T$ depends on the merging neighborhood.  The matrix $T$ is circulant 
%due to the periodic boundary conditions.  
%The first row of the $T$ matrix associated with central merging tile of 
%size $2n+1$ is
%$$
%\frac{1}{2n+1}[\underbrace{1, 1, \hdots,1}_{n+1}, 0, \hdots, 0,\underbrace{1, 1, \hdots, 1}_{n}].
%$$
%The final step of the SRD method maps the averages on the merging tiles back to the original mesh with the matrix $S$, i.e.,
% \begin{align}
%   U^{n+1}_i =  \frac{1}{J} \sum^n_{k = - n}\hat Q_{i+k} &- (\hat Q_{i+ k + n + 1}-\hat Q_{i- k - n-1})\frac{k}{J+1}\\
%   &+ (\hat Q_{i+k+n + 1} -2 \hat Q_{i+k} +\hat Q_{i-k - n-1})\left(2\frac{k^2}{(J+1)^2} - \frac{n}{3(J+1)}\right).
% \end{align}
The final update for the finite volume method without slope reconstruction is
$$
U^{n+1} = S(I + \Delta t D)U^n.
$$
where the matrix $S$ represents the SRD operation. 
%If a high-order reconstruction in the finite volume method is used, 
%we use an SSP-RK method of the same order.  Since SSP-RK methods are 
%convex combinations of forward Euler time steps, we apply the SRD method 
%after each forward Euler step. 
For linear reconstructions, the final update is
\begin{equation}
\begin{aligned}
    U^{(1)} &= S(I + \Delta t D)U^n, \\
    U^{(2)} &= S(I + \Delta t D)U^{(1)},\\
    U^{n+1} &= \frac{1}{2}U^n + \frac{1}{2}U^{(2)},
\end{aligned}
\end{equation}
%and 
%\begin{equation}
%\begin{aligned}
%    U^{(1)} &= ST(I + \Delta t L)U^n, \\
%    U^{(2)} &=  \frac{3}{4}U^n + \frac{1}{4}ST(I+\Delta t L)U^{(1)} ,\\
%    U^{(3)} &= ST(I+\Delta t L)U^{(2)}, \\
%    U^{n+1} &= \frac{1}{3}U^n + \frac{2}{3}U^{(3)},
%\end{aligned}
%\end{equation}
%respectively.  
The magnitude of the dominant eigenvalue $|\lambda_{\text{max}}|$ of the complete scheme is a function of $\Delta t$.  We compute the maximum $\Delta t$ for which $|\lambda_{\max}| \leq 1$.  In Figure \ref{fig:cflvsneigh}, the CFL number is given by $\text{CFL} = a\frac{\Delta t}{\Delta x}$.  
We observe that the maximum allowable time step increases with the size of the merging neighborhood.
\begin{figure}
	\centering
	\includegraphics[width=0.45\linewidth]{figs/cfl_vs_neigh}
	\caption{CFL number versus neighborhood size.  As the merging neighborhood increases, the maximum allowable time step increases as well.}
	\label{fig:cflvsneigh}
\end{figure}

%\section{Nonuniform grid}
 
%  \subsubsection*{First order method}
% The finite volume update is
% $$
% \hat U_i = U^n_i -a\frac{\Delta t}{h}(U^n_i - U^n_{i-1}),
% $$
% or, in matrix form
% $$
% \hat U = \left(I -\Delta t L \right)U^n,
% $$
% where the matrix $L$ is circulant, and its first row is
% $$
% \frac{a}{h}[1, 0, \hdots, 0, -1].
% $$
% The final solution update is
% $$
% U^{n+1}_i = \frac{1}{N_i}\sum_{k \in W_i} \hat Q_k,
% $$
% where $W_k$ is the set of merging tiles to which cell $i$ belongs.  In matrix form, this becomes
% \begin{equation}
%     U^{n+1} = S_0 \hat Q,
% \end{equation}
% where the matrix $S_0$ is circulant.  
% The first row of the $S_0$ matrix associated with the central merging tiles is
% $$
% [\underbrace{\frac{1}{2n+1}, \frac{1}{2n+1}, \hdots, \frac{1}{2n+1}}_{n+1}, 0, \hdots, 0,\underbrace{\frac{1}{2n+1}, \hdots, \frac{1}{2n+1}}_{n}],
% $$
% where $2n+1$ is the number of cells in the central merging neighborhood, since cell $i$ belongs to merging tiles to the left and right.
% The entire state redistribution algorithm is
% $$
% U^{n+1} = S_0T \left(I + a\frac{\Delta t}{h} L \right)U^n.
% $$


% % We aim to determine how large of a time step we can take in terms of the size of the merging neighborhood.
% % In Figure \ref{fig:cfl_vs_nsize}, we plot the maximum allowable CFL number for two discretizations with $N=500$ and $N=1000$ cells using the central merging neighborhoods of size $J$, i.e., we plot the ratio of the maximum stable $\Delta t$ over the size of the merging neighborhood $Jh$: $\text{CFL} = \frac{\Delta t_{\text{max}}}{Jh}$.



%  \subsubsection*{Second order method}
% The finite volume update is
% $$
% \hat U_i = U^n_i -a\frac{\Delta t}{h} \left(U^n_i +\frac{U^n_{i+1} - U^n_{i-1}}{4} - U^n_{i-1}-\frac{U^n_{i} - U^n_{i-2}}{4} \right),
% $$
% $$
% \hat U_i = U^n_i -a\frac{\Delta t}{h} \left(\frac{1}{4}U^n_{i-2} -\frac{5}{4}U^n_{i-1} + \frac{3}{4}U^n_i + \frac{1}{4}U^n_{i+1} \right),
% $$
% or, in matrix form
% $$
% \hat U = \left(I -\Delta t L \right)U^n,
% $$
% where the matrix $L$ is circulant, and its first row is
% $$
% \frac{a}{h}\left[\frac{3}{4}, \frac{1}{4}, \underbrace{0, \hdots, 0}_{N-4},\frac{1}{4}, -\frac{5}{4} \right].
% $$
% The reconstruction on the merging neighborhood is
% \begin{equation}\label{eq:q_reconstruction}
% \hat q_i(x) = \hat Q_i + \frac{\hat Q_{i+r}-\hat Q_{i-r}}{(2r+1)h} (x - x_i),
% \end{equation}
% % where $\frac{\hat \sigma_i}{V_i} = \frac{\hat Q_{i+n + 1}-\hat Q_{i-n-1}}{X_{i+n + 1}-X_{i - n - 1}}$.  
% We reconstruct the slope on merging tile $i$ using merging tiles $i+r$ and $i-r$, with $r \geq 1$.
% On a structured mesh, the weighted centroid of merging tile $i$ becomes the standard centroid of cell $i$, i.e., $\hat{x}_i = x_i$.  
% % Additionally, the distance $\hat{x}_{f}-\hat{x}_{b} = (J+1)h$.
% % Therefore, \eqref{eq:q_reconstruction} with $\hat \sigma_i$ defined in \eqref{eq:sigma} becomes
% % $$
% % \hat q_i(x) = \hat Q_i + \frac{\hat Q_{i+n + 1}-\hat Q_{i-n-1}}{(J+1)h} (x - x_i).
% % $$
% The final solution update is
% \begin{equation}\label{eq:final_update_q_temp}
%     U^{n+1}_i = \frac{1}{J} \sum^n_{k = - n} \frac{1}{h}\int^{x_{i+\frac{1}{2}}}_{x_{i-\frac{1}{2}}}\hat q_{i+k}(x) ~dx .
% \end{equation}
% Due to the linearity of $\hat q_{i+k}(x)$, the above becomes
% \begin{equation}\label{eq:final_update_q}
%     U^{n+1}_i = \frac{1}{2n+1} \sum^n_{k = - n} \hat q_{i+k}(x_i).
% \end{equation}
% Evaluating the merging tile $i+k$'s reconstruction at $x_i$, we have
% \begin{equation}\label{eq:q_k}
%     \hat q_{i+k}(x_i) = \hat Q_{i+k} - (\hat Q_{i+k+r}-\hat Q_{i+k-r})\frac{k}{2r+1}.
% \end{equation}
% In matrix form, one forward Euler time step is
% \begin{equation}
%     U^{n+1} = (S_0+S_1) \hat Q,
% \end{equation}
% where the matrix $S_1$ is circulant.  With RK2 time stepping, the final


% \subsubsection*{Time step restriction versus merging neighborhood size}
% In this section we examine the maximum stable CFL number in terms of the size of the merging neighborhood. 

%\begin{figure}
%    \centering
%\subfloat[CFL versus number of cells in merging tile for various orders of approximation ($p$).]{\includegraphics[width=0.45\linewidth]{figs/cflvsj.pdf}} \hfill
%\subfloat[CFL versus number of cells in merging tile for various reconstruction stencil widths ($k$).]{\includegraphics[width=0.45\linewidth]{figs/cflvsk.pdf}}
%    \caption{Influence of the merging neighborhood size and the merging tile size on the CFL number.}
%    \label{fig:cflvsj}
%\end{figure}



%\begin{figure}
%    \centering
%    \includegraphics{figs/consistency.pdf}
%    \caption{}
%    \label{fig:consistency}
%\end{figure}

%\subsection{GKS stability}
%We can also look at stability at a boundary, in a problem that is closer
%to the 2D case. 
%
%
%SHOULD WE MENTION FRD IS UNSTABLE on UNIFORM 1D CASE
%
%SHOULD WE MENTION NOT MONOEONTE EVEN IN 1st ORDER CASE
