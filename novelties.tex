\documentclass{article}
\usepackage[utf8]{inputenc}

\title{Novelties}
\date{}

\begin{document}

\maketitle

The purpose of this research is to develop a simple and robust algorithm 
to solve the small cell problem on cut cell meshes.
Our algorithm, called \textit{state redistribution}, stabilizes second order 
finite volume schemes on cut cell meshes and allows the use of a time step 
proportional to the size of a full cell.  
This is a novel technique that only requires cell centroids, volumes, and 
standard connectivity information already available in many cut cell codes.

State redistribution is applied to the numerical solution after each step 
or stage of an explicit time stepping scheme.
The cut cells are temporarily merged into 
larger, possibly overlapping, neighborhoods using a weighted convex
combination of neighboring cells.
Next, the solution on the cut cells is replaced with a stabilized 
value determined from the nearby merging neighborhoods.
Our approach is linearity preserving, conservative, and does not reduce the accuracy of the underlying finite volume method.

\end{document}
